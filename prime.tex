\documentclass[12pt]{article}
\usepackage[utf8]{inputenc}
\usepackage{graphicx}
\usepackage{enumitem,amssymb}
\usepackage{tabularx}
\usepackage{calc}
\usepackage{cprotect}
\usepackage{xcolor}
\usepackage[a4paper,textwidth=16cm,top=2cm,bottom=2cm,headheight=25pt,headsep=12pt,footskip=25pt]{geometry}
\usepackage{fancybox}
\newenvironment{framed}%
  {\begin{Sbox}\begin{minipage}{\dimexpr\linewidth-2\fboxrule-2\fboxsep}}%
  {\end{minipage}\end{Sbox}\fbox{\TheSbox}}

\usepackage{pifont}
\usepackage{ifthen}

\newcommand{\corrige}{0}
\newcounter{possibility}

\newcommand{\correct}{%
  \addtocounter{possibility}{1}
  \ifthenelse{\equal{\corrige}{0}}%
    {\item[\ding{\the\numexpr\value{possibility}}]}%
    {\item[\textcolor{red}{\ding{\the\numexpr\value{possibility}-10}}]}%
}

\newcommand{\leurre}{%
  \addtocounter{possibility}{1}
  \item[\ding{\the\numexpr\value{possibility}}]%
}

\newcommand{\enonce}[1]{%
  \noindent
  Question~:
  \vspace*{.5\baselineskip}

  \noindent
  \begin{framed}
    #1
  \end{framed}
}

\newcommand{\difficulte}[1]{%
  Difficulté : 
  \ifthenelse{#1 > 1 \or #1 = 1}{\ding{72}}{\ding{73}} 
  \ifthenelse{#1 > 2 \or #1 = 2}{\ding{72}}{\ding{73}}
  \ifthenelse{#1 > 3 \or #1 = 3}{\ding{72}}{\ding{73}}
  \ifthenelse{#1 > 3}{S}{}%
}

\newcommand{\possibilites}[1]{
  \setcounter{possibility}{171} 
  \vspace*{\baselineskip}
  \begin{itemize}[label={}]
    #1
  \end{itemize}
}


\newcommand{\pourquoi}[1]{}

\begin{document}
\section{}
\difficulte{1}

\enonce{Que représente le premier paramètre (souvent appelé \verb|self|) dans une méthode codée en python ?}
\possibilites{
    \leurre la classe à laquelle la méthode appartient
    \correct l'objet qui a appelé la méthode
    \leurre c'est un paramètre comme un autre sans signification particulière
}
\pourquoi{}

\section{}
\difficulte{1}

\enonce{Je permets à un objet d'être différent d'un autre objet de la même classe. Je suis~:}
\possibilites{
    \correct un attribut
    \leurre une methode
    \leurre un {m design patern}
}
\pourquoi{}

\section{}
\difficulte{1}

\enonce{Qui est le plus important entre le main, les tests et les fonctions?}
\possibilites{
    \correct Le main
    \correct Les tests
    \correct Les fonctions
}
\pourquoi{}


\end{document}