\documentclass[12pt]{article}
\usepackage[utf8]{inputenc}
\usepackage{graphicx}
\usepackage{enumitem,amssymb}
\usepackage{tabularx}
\usepackage{calc}
\usepackage{cprotect}

\usepackage{xcolor}

\usepackage[%
      a4paper,%
      textwidth=16cm,%
      top=2cm,%
      bottom=2cm,%
      headheight=25pt,%
      headsep=12pt,%
      footskip=25pt]{geometry}%
      

\usepackage{fancybox}
\newenvironment{framed}%
       {\begin{Sbox}\begin{minipage}{\dimexpr\linewidth-2\fboxrule-2\fboxsep}}%
       {\end{minipage}\end{Sbox}\fbox{\TheSbox}}


\usepackage{pifont}

\newcommand{\corrige}{0}

\usepackage{ifthen}

\newcounter{possibility}

\newcommand{\correct}{%
\addtocounter{possibility}{1}
\ifthenelse{\equal{\corrige}{0}}%
    {\item[\ding{\the\numexpr\value{possibility}}]}%
    {\item[\textcolor{red}{\ding{\the\numexpr\value{possibility}-10}}]}%
}

\newcommand{\leurre}{%

\addtocounter{possibility}{1}
\item[\ding{\the\numexpr\value{possibility}}]%
}

\newcommand{\enonce}[1]{%
\noindent
Question~:
\vspace*{.5\baselineskip}

\noindent
\begin{framed}
    #1
\end{framed}}

\newcommand{\difficulte}[1]{%
Difficulté : 
\ifthenelse{#1 > 1 \or #1 = 1}{\ding{72}}{\ding{73}} 
\ifthenelse{#1 > 2 \or #1 = 2}{\ding{72}}{\ding{73}}
\ifthenelse{#1 > 3 \or #1 = 3}{\ding{72}}{\ding{73}}
\ifthenelse{#1 > 3}{S}{}%
}


\newcommand{\possibilites}[1]{%


\vspace*{\baselineskip}


\noindent
Cochez l'unique bonne réponse~:
\vspace*{.5\baselineskip}

\noindent
\begin{framed}  
    \begin{enumerate}
        \setcounter{possibility}{191}
        #1
    \end{enumerate}
\end{framed}}%

\newcommand{\texte}[1][]{%


\vspace*{\baselineskip}


\noindent
Écrivez votre réponse~:
\vspace*{.5\baselineskip}

\noindent
\begin{framed}
    \hfill
    \ifthenelse{\equal{#1}{}}{\vspace*{1cm}}{\vspace*{#1}}    
\end{framed}}%


\newcommand{\pourquoi}[1]{%
\ifthenelse{\equal{\corrige}{1}}{%

\vspace*{\baselineskip}


\noindent
\textcolor{red}{Élement de solution~:}
\vspace*{.5\baselineskip}

\noindent
\begin{framed}
\textcolor{red}{%
#1}
\end{framed}}{}%
}


\newcommand{\chapeau}[1]{%


\vspace*{\baselineskip}


\noindent
\begin{framed}
    #1
\end{framed}}%


% @@@@@@@@@@@@@@@@@@@@@@@@@ uncomment pour le corrigé 

\renewcommand{\corrige}{1}

% @@@@@@@@@@@@@@@@@@@@@@@@@

       
\begin{document}

\begin{center}
  \begin{tabular}{c}
  \hline\\%\vspace{0.1cm}
  {\textsc{\'Ecole Centrale Marseille}}\vspace{0.1cm}
  \\
%  
    {\bf {\Large Modélisation et Conception Objet}}\\%\vspace{0.2cm}
    \\
    {\bf  { Contrôle 1A }}\\
    {\footnotesize 27/06/2107}\\
    \hline
  \end{tabular}
\end{center}
\vspace{0.6cm}

\noindent
{\em Les réponses sont à donner directement sur le sujet. Un espace est réservé pour chaque réponse.}

\vspace*{1cm}
\noindent
\begin{tabular}{|l|p{10cm}|}
    \hline
    Nom : & \\
    \hline
    Prénom : & \\
\hline
    Numéro carte d'étudiant : & \\
    \hline
    Année d'étude :& \\    
    \hline
    Est-ce un rattrapage ? :& \\    
    \hline
    Signature : & \vspace*{2cm}\\
    \hline
    
\end{tabular}


% ### Debut version sujet ###################################
\noindent
\\
Version du sujet : {\small AZATHOTH}\\
% ### Fin version sujet ###################################


\vspace*{2cm}
\noindent
Barème. Pour chaque question : 
\begin{itemize}
    \item ne pas répondre donne 0 point,
    \item répondre de façon exacte donne : $\mathbf{+\frac{2}{3}}$ points
    \item répondre de façon inexacte\footnote{le nombre de points négatifs varie pour que si l'on répond de façon aléatoire l'espérance soit nulle} : 
    \begin{itemize}
        \item pour une réponse de type VRAI/FAUX : $\mathbf{-\frac{2}{3}}$ points,
        \item pour une réponse libre : $\mathbf{-\frac{2}{3}}$ points,
        \item pour une réponse de type 1 parmi 3 : $\mathbf{-\frac{1}{3}}$ points.
    \end{itemize}
\end{itemize}


\vspace*{1cm}
\noindent
{\bf Les questions sont placées dans un ordre aléatoire et sont indépendantes. Il y a des questions simples et d'autres plus complexes, n'hésitez pas à lire tout le sujet avant de commencer.}
\vspace*{2cm}




\newpage

% ### Debut questions random ###################################
% ### Question ###################################
\vbox{%
\section{}

\difficulte{1}

\enonce{
Je permets à un objet d'être différent d'un autre objet de la même classe. Je suis~:}

\possibilites{
% ### begin random response
    \correct un attribut
    \leurre une méthode
    \leurre un {\em design pattern}
% ### end random response     
}

\pourquoi{Un objet est composé de méthodes et d'attributs et les méthodes sont communes à tous les objet d'une même classe.}
}


% ### Question ###################################
\vbox{%
\section{}

\difficulte{1}

\enonce{
Une méthode privée peut être utilisée par une autre méthode définie dans la même classe.}

\possibilites{
    \correct vrai
    \leurre faux
}

\pourquoi{Privé est défini par rapport à ce qui est en dehors de la classe}
}


% ### Question ################################### 
\vbox{%
\section{}

\difficulte{1}

\enonce{
Une méthode privée peut être utilisée par un attribut défini dans la même classe.
}
\possibilites{
    \leurre vrai
    \correct faux
}

\pourquoi{Un attribut est une variable, il n'utilise rien}
}


% ### Question ################################### 
\section{}
\chapeau{
Les deux classes de la figure~\ref{fig:store_uml} sont liées.     
}


\vbox{%
\subsection{}
\difficulte{2}
\enonce{
Ce lien est :
}

\cprotect\possibilites{
% ### begin random response
    \leurre Une dépendance de \verb|StateObject| pour \verb|Store|
    \correct Une dépendance de \verb|Store| pour \verb|StateObject|
    \leurre Une dépendance mutuelle des deux classes
% ### end random response    
}
\cprotect\pourquoi{Un attribut de \verb|Store| est de type \verb|StateObject|.}
}


\vbox{%
\subsection{}
\difficulte{2}
\enonce{
Ce lien est appelé~: 
}

\possibilites{
% ### begin random response    
    \correct Agrégation
    \leurre Composition
    \leurre Héritage
% ### end random response         
}

\cprotect\pourquoi{ C'est une agrégation car l'objet de type \verb|StateObject| dans \verb|Store| est importé à l'init il n'est pas créé par \verb|Store|}

}


% ### Question ###################################
\section{}

\cprotect\chapeau{
On utilise les deux classes de la figure~\ref{fig:store_uml}. Trois Propositions de code sont données~:
\begin{enumerate}
    \item \begin{minipage}{5cm}
        \begin{verbatim}
            state = StateObject()
            state.set_state(1)
            store = Store(state)
            store.saved_state = state.get_state()
        \end{verbatim}
    \end{minipage}
    \item \begin{minipage}{5cm}
            \begin{verbatim}
                state = StateObject()
                state.set_state(1)
                store = Store(state)
            \end{verbatim}
        \end{minipage}
    \item \begin{minipage}{5cm}
            \begin{verbatim}
                state = StateObject(1)
                store = Store(state)
            \end{verbatim}
        \end{minipage}        
\end{enumerate}
}

\vbox{%
\subsection{}
\difficulte{2}
\enonce{
    Une seule des 3 propositions est correcte par rapport au schéma UML. Laquelle~?}

\cprotect\possibilites{
    \leurre proposition~1
    \correct proposition~2
    \leurre proposition~3
}

\cprotect\pourquoi{L'init \verb|deStateObject| n'a pas de paramètre et l'attribut \verb|saved_state| de \verb|Store| est privé.}
}

\vbox{%
\subsection{}
\difficulte{3}
\enonce{
    Une seule des 3 propositions est incorrecte en python. Laquelle~?
}

\cprotect\possibilites{
    \leurre proposition~1
    \leurre proposition~2
    \correct proposition~3
}
\pourquoi{La notion de variable privée n'existe pas en python.}
}


% ### Question ###################################
\vbox{%
\section{}

\difficulte{1}

\cprotect\enonce{
Que représente le premier paramètre (souvent appelé \verb|self|) dans une méthode codée en python ?
}
\possibilites{
% ### begin random response    
    \leurre la classe à laquelle la méthode appartient 
    \correct l'objet qui a appelé la méthode
    \leurre c'est un paramètre comme un autre sans signification particulière
% ### end random response             
}

}


% ### Question ###################################
\vbox{%
\section{}

\difficulte{1}

\cprotect\enonce{
En UML, la description d'une méthode sans paramètre doit posséder des parenthèses (eg: \verb|ma_methode()|)
}

\possibilites{
    \correct vrai
    \leurre faux
}
}


% ### Question ###################################
\vbox{%
\section{}

\difficulte{2}

\cprotect\enonce{
En python, pour la méthode définie telle que : 
\begin{verbatim}
def ma_methode(self, param=0):
    # corps de la methode    
\end{verbatim}
}

\cprotect\possibilites{
% ### begin random response    
    \leurre \verb|param| vaut 0 dans tous les cas
    \correct la valeur par défaut de \verb|param| est 0
    \leurre la méthode n'a pas de paramètre
% ### end random response     
}
}


% ### Question ###################################
\vbox{%
\section{}

\difficulte{3}

\cprotect\enonce{
En python, pour la classe définie telle que : 
\begin{verbatim}
class GreenCarpet:
    def __init__(self):
        self.dices = tuple(Dice() for i in range(5))
\end{verbatim}
La relation en \verb|GreenCarpet| et \verb|Dice| est :
}

\possibilites{
% ### begin random response 
    \leurre un héritage
    \leurre une agrégation
    \correct une composition
% ### end random response 
}

\cprotect\pourquoi{Les objets de type \verb|Dice| sont créés dans le constructeur de \verb|GreenCarpet|.}
}



% ### Question ###################################
\vbox{%
\section{}

\difficulte{4}

\cprotect\enonce{
En python, pour la classe définie telle que : 
\begin{verbatim}
class DiceStat(Dice):
    def __init__(self, value=1):
        super().__init__(value)
        self.stat = dict()
\end{verbatim}
La ligne commençant par \verb|super| signifie :
}

\cprotect\possibilites{
% ### begin random response 
    \leurre un appel à la méthode \verb|__init__| pour un objet de type super
    \correct un appel à la méthode \verb|__init__| de la classe \verb|Dice|
    \leurre un appel récursif à la méthode \verb|__init__| que l'on est entrain de définir.
% ### end random response     
}

\cprotect\pourquoi{\verb|super()| permet d'acceder à l'attribut/méthode à droite du ``.'' pour la classe mère de l'objet, ici \verb|Dice|.}
}


% ### Question ###################################
\section{}

\chapeau{On considère le diagramme UML de la figure~\ref{fig:student_uml}}

\vbox{%
\subsection{}
\difficulte{3}

\enonce{Liens entre les trois classes~:
}

\cprotect\possibilites{
% ### begin random response
    \leurre la classe \verb|Student| hérite de la classe \verb|Person| et la classe \verb|Group| hérite de la classe \verb|Student|
    \leurre il y a une relation d’association entre la classe \verb|Person| et la classe \verb|Student| (une agrégation ou une composition) et la classe \verb|Group| hérite de la classe \verb|Student|
    \correct la classe \verb|Student| hérite de la classe \verb|Person|  et il y a une relation d’association (une agrégation ou une  composition) entre la classe \verb|Group| et la classe \verb|Student| 
% ### end random response         
}
}

\vbox{%
\subsection{}
\difficulte{2}

\cprotect\enonce{pour une instance de la classe \verb|Student|
}

\cprotect\possibilites{
% ### begin random response
    \leurre l'\verb|id| est uniquement décrite par son \verb|ine_number|
    \correct l'\verb|id| est uniquement décrite par son \verb|name| et son \verb|firstname|
    \leurre l'\verb|id| est uniquement décrite par son \verb|ine_number|, son \verb|name| et son \verb|firstname|
% ### end random response     
}

\cprotect\pourquoi{\verb|id| est privée et uniquement défini dans la classe \verb|Person|, elle ne peut donc a priori pas être modifiée dans \verb|Student|}
}


% ### Question ###################################
\vbox{%
\section{}
\difficulte{1}

\cprotect\enonce{Le nom \verb|saved_state| de la  classe \verb|Store| du diagramme de la figure~\ref{fig:store_uml} est :
}

\cprotect\possibilites{
% ### begin random response
    \leurre  une méthode 
    \correct  un attribut
    \leurre  une fonction
% ### end random response    
}
}


% ### Question ###################################
\section{}

\cprotect\chapeau{La figure~\ref{fig:observer_uml} montre le diagramme UML du design pattern {\em Observer}. Les objets de la classe \verb|Observer| s'inscrivent à un sujet (via la méthode \verb|add|) pour être prévenus de chaque changement d'état (via la méthode \verb|notify|). Tous les codes demandés sont à écrire en python ou en Java si vous êtes en 2A.}

\vbox{%
\subsection{}
\difficulte{1}

\cprotect\enonce{Quel est la fonction de ce couple de classe ?
}

\possibilites{
% ### begin random response
    \leurre permet aux objets observant un sujet de le modifier à tout moment
    \correct permet aux objets observant un sujet d'être informé des changements d'un sujet par celui-ci
    \leurre permet au sujet de modifier les objets qui l'observent 
% ### end random response    
}
}

\vbox{%
\subsection{}
\difficulte{2}

\cprotect\enonce{Ecrivez la méthode \verb|add| de la classe \verb|Subject|, qui ajoute l'objet passé en paramètre à la liste des éléments observant le sujet.
}
\texte[3cm]

\cprotect\pourquoi{
\begin{verbatim}
def add(self, observer):
    self.oberver_list.append(observer)
\end{verbatim}
}
}

\vbox{%
\subsection{}
\difficulte{3}

\cprotect\enonce{Ecrivez la méthode update de la classe \verb|Observer| qui place dans \verb|subject_state| l'état courant d'objet subject de la classe \verb|Subject| passé en paramètre.
}

\texte[3cm]
\cprotect\pourquoi{
\begin{verbatim}
def update(self, subject):
    self.subject_update = subject.get_state()
\end{verbatim}
}

}


\vbox{%
\subsection{}
\difficulte{4}

\cprotect\enonce{Ecrivez la méthode \verb|notify| de \verb|Subject| qui permet la mise à jour des éléments observant le \verb|Subject|
}
\texte[3cm]
\cprotect\pourquoi{
\begin{verbatim}
def notify(self):
    for observer in self.observer_list:
        observer.update(self)
\end{verbatim}
}

}


% ### Question ###################################
\vbox{%
\section{}
\difficulte{1}

\cprotect\enonce{Quand fait-on un appel à une méthode \verb|__init__| ?

}
\possibilites{
% ### begin random response
    \leurre lors de la création d'une classe
    \correct lors de la création d'une instance de la classe
    \leurre lors de la création d'un attribut
% ### end random response    
}
}


% ### Question ###################################
\vbox{%
\section{}
\difficulte{1}

\cprotect\enonce{Comment permettre à un attribut privé d'un objet d'être modifié ``en dehors'' de celui-ci
}
\possibilites{
% ### begin random response
    \leurre en créant un {\em getter}
    \leurre c'est impossible
    \correct en créant un {\em setter}
% ### end random response
}
}


% ### Question ###################################
\vbox{%
\section{}
\difficulte{2}

\enonce{Donnez le diagramme UML d’une classe {\em A} ayant les propriétés suivantes :
\begin{itemize}
    \item un attribut $att_1$, de type réel, dont l’accès est protégé et qui a comme valeur initiale $3.14$. 
    \item un attribut $att_2$, librement accessible qui est une série de chaînes de caractères. 
    \item une méthode $delta$ de visibilité protégée qui utilise un entier $n$ (non modifié), un objet 
$O$ de la classe {\em B} qu’elle modifie, et qui retourne une valeur booléenne. On ne représentera pas la classe {\em B} sur le diagramme. 

\end{itemize} 

}
\texte[5cm]
}


% ### Question ###################################
\vbox{%
\section{}
\difficulte{3}

\enonce{Sur un ordinateur, les fichiers sont organisés en {\it répertoires}. Un répertoire est {\it quelque chose} qui contient des fichiers (élémentaires) ou d'autres répertoires.
Représenter la notion de répertoire par un diagramme UML.
}
\texte[5cm]
}


% ### Question ###################################
\vbox{%
\section{}
\difficulte{4}

\cprotect\enonce{Si \verb|C| est une classe, donnez la différence entre \verb|c = C| et \verb|c = C()|
}
\texte[5cm]
\pourquoi{La première affectation affecte la classe à une variable, la seconde le résultat de l'``exécution'' de la classe, c'est à dire un objet.
}
}


% ### Question ###################################
\vbox{%
\section{}
\difficulte{2}

\cprotect\enonce{Complétez la classe \verb|Student| de la figure~\ref{fig:student_uml} pour que l'on puisse connaitre et modifier la dernière note d'informatique d'un étudiant.
}
\texte[5cm]

\pourquoi{on peut soit mettre uniquement un attribut public, soit mettre un setter/getter plus un attribut privé}

}


% ### Question ###################################
\vbox{%
\section{}
\difficulte{1}

\enonce{Comment représenter en UML le lien d'agrégation ?
}
\possibilites{
% ### begin random response
    \correct un trait plein et un losange vide 
    \leurre un trait plein et une étoile
    \leurre un trait pointillé et une étoile
% ### end random response
}
}

% ### Question ###################################
\section{}
\vbox{%
\subsection{}
\difficulte{3}

\cprotect\enonce{Proposez le diagramme UML d'une classe \verb|Conversion| dont l'initialisation prend 3 paramètres :
\begin{itemize}
    \item la monnaie à convertir,
    \item la monnaie dans laquelle convertir,
    \item un réel représentant le taux de conversion (1 unité de la monnaie à convertir correspond à {\em taux}
unité de la devise dans laquelle convertir).
\end{itemize}
Cette classe doit avoir une méthode permettant de connaître le nombre d'unité de la monnaie dans laquelle convertir à partir d'un nombre d'unité de la monnaie à convertir.

}
\texte[5cm]
}

\vbox{%
\subsection{}
\difficulte{4}

\enonce{Implémentez la classe en python (ou en Java si vous êtes en 2A)
}
\texte[5cm]
}


% ### Question ###################################
\vbox{%
\section{}
\difficulte{4}

\enonce{A quoi pourrait bien servir le design pattern de la figure~\ref{fig:strategy_uml}~?
}
\texte[5cm]

\pourquoi{C'est le pattern strategy. Plusieurs implémentation d'un même problème (algorithme de tris différent par exemple).}
}


% ### Question ###################################
\vbox{%
\section{}
\difficulte{3}

\enonce{On considère une classe Point permettant de créer des points du plan à partir de leurs deux coordonnées : x et y. cette classe contient une méthode add() réalisant l’addition de ces deux points. Comment doit-on décrire cette méthode en UML?
}

\possibilites{
% ### begin random response
    \correct + add(point: Point): Point
    \leurre + add(self, point : Point): Point
    \leurre + add(x : double, y: double): Point
% ### end random response
}

}


% ### fin questions random ###################################

% ################################### DERNIERE QUESTION ###################################

% \vbox{%
% \section{}
%
% \difficulte{0}
%
% \enonce{
% Que signifie le sigle MCO pour vous ?}
%
% \possibilites{
%     \correct Mouloudia Club Oranais
%     \correct Maintien en Condition Opérationnelle
%     \correct Menuiserie de la Côte Ouest
%     \leurre Autre chose que j'explicite dans le texte
% }
% \texte[1.5cm]
% \pourquoi{Les trois explications sont autentiques.}
% }


\newpage
\section*{Annexes}

\subsection*{Symbole diagramme UML}

La figure~\ref{fig:UmlDiag} montre les caractéristiques attendues d'un diagramme UML. On rappelle qu'une instance d'une classe est un objet dans le type est cette classe.

Pour les diagrammes UML de cet examen, le type des variables n'est indiqué que s'il est important pour la compréhension. La plupart du temps, seul le nom de la variable, de l'attribut ou de la méthode est mentionné.


\begin{figure}[!h]
    \centering
    \includegraphics[scale=.7]{uml_attributs.png}        

    \caption{Un diagramme UML.\label{fig:UmlDiag}}
\end{figure}


\subsection*{Classes du contrôle}


\begin{figure}[!h]
    \centering
    \includegraphics[scale=.6]{uml_store_state.png}        

    \cprotect\caption{Le diagramme UML des classes \verb|Store| et \verb|StateObject|.\label{fig:store_uml}}
\end{figure}


\begin{figure}[!h]
    \centering
    \includegraphics[scale=.6]{uml_student.png}        

    \cprotect\caption{Le diagramme UML des classes \verb|Person|, \verb|Student| et \verb|Group|.\label{fig:student_uml}}
\end{figure}


\begin{figure}[!h]
    \centering
    \includegraphics[scale=.6]{uml_observer.png}        

    \cprotect\caption{Le diagramme UML des classes \verb|Subject| et \verb|Observer|.\label{fig:observer_uml}}
\end{figure}



\begin{figure}[!h]
    \centering
    \includegraphics[scale=.6]{uml_strategy.png}        

    \cprotect\caption{Un design pattern bien utile.\label{fig:strategy_uml}}
\end{figure}
\end{document}
